\documentclass[12pt]{article}
\usepackage{amsmath}
\usepackage{amssymb}
\usepackage{latexsym}
\usepackage{graphicx}
\usepackage{mathrsfs}
\usepackage{CJKutf8}
\author{qit}
\title{暑期培训——网络编程\&进程间通信预习/作业}
\begin{document}
\begin{CJK}{UTF8}{gbsn}
\maketitle
\section{配置}
\paragraph{}有python3.6即可,最好有linux系统。

\section{网络编程预习}
\paragraph{}网络编程1课时,7.6第2节。
\paragraph{}基本没有预习任务。

\section{进程间通信}
\paragraph{}进程间通信2课时,7.7第1、2节课。
\paragraph{预习}廖雪峰python教程博客中的下列内容:
\\预习进程和线程前言、进程vs线程(对进程和线程没有概念必看)
\\预习多进程、多线程(选看)

\section{网络编程\&进程间通信作业}
\paragraph{}网络编程和进程不单独布置作业,合在一起布置。
\paragraph{1.}自行练习廖雪峰博客上多线程多进程部分和网络编程部分的demo。
\paragraph{2.}实现一个简单的群聊软件,利用C/S设计模式,使得所有链接到一个server上的client都能加入到同一个对话中,基本功能是每一个加入的client有一个名字,之后每一个client发送一句话,其他client都能收到。无需GUI,只需要在bash下实现即可。
\\难度偏低,主要是熟悉多线程和socket编程,不用提交。
\section{作业3-实现简单的游戏平台}
\subsection{游戏描述}
\paragraph{}一个游戏进行100轮,共有两名博弈的玩家。每个玩家初始的时候有1000个硬币,在每一轮,每名玩家在不知道对方有几枚硬币的情况下,下注自己的硬币,并由游戏的服务器进行对比,胜利者无论双方硬币的差额是多少,统一获得2积分,失败者获得0积分,如果平局,则各获得1积分,最后积分高者获胜。
\subsection{设计要求}
\paragraph{1.}利用C/S模式进行设计,游戏服务器端使用python实现,游戏的两个客户端利用client实现,并进行跨平台的进程间通信。
\paragraph{2.}在游戏的每一轮开始开始计时,每位玩家只有最多1s的计算时间,必须在1s内下注,并发送给服务器,否则此次视为下注为0。
\paragraph{3.}游戏的每一轮结束,服务器拿到两位玩家的下注进行计算,并更新积分,并且把此次对方的下注情况和当前回合数目发送给玩家。并且需要进行检查,如果一位玩家此次下注数目超过剩余硬币数目,则视为下注为0。
\paragraph{4.}在游戏结束后,发送胜利方给客户端。
\subsection{重点}
\paragraph{1.}注意在有一方超时的情况下,不要发生死锁。
\paragraph{2.}注意通信的格式,可以采用json/protobuf库进行实现。
\paragraph{3.}游戏是我乱编的,不需要设计一个玩这个游戏的算法,但需要写一个每一轮以$p_1$概率在$0-1000$随机发送一个数;以$p_2$概率先sleep一次,sleep时间长度服从0.5~10s的均匀分布,之后再随机发送一个数;以$p_3$概率进入死循环。($p_1+p_2+p_3=1$,大小可以任取)


\section{作业提交}
\paragraph{}多线程和socket编程的混合编程是平台组最重要的地方,因此每一个想要加入ts平台组的同学都应该完成上述需要提交的作业。
\paragraph{提交方式}qit16@mails.tsinghua.edu.cn   齐涛 电子系科协软件部




\end{CJK}
\end{document}